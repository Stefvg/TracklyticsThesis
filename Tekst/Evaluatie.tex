\chapter{Evaluatie}
In deze sectie wordt de ontwikkelde library ge\"evalueerd. Allereerst wordt er gekeken naar wat de impact van de library op de performance van de applicatie is. Verder in dit hoofdstuk wordt de schaalbaarheid van de library bekeken. Ten slotte wordt er nog gekeken hoeveel effort er van de developer nodig is om de library te integreren in een applicatie.


\section{Performance}
In het verloop van deze thesis is al enkele keren aan bod gekomen dat de prestaties van een applicatie \'e\'en van de belangrijkste eigenschappen van een applicatie is. De impact op de performance van de library op de performance moet aanvaardbaar zijn en zo klein mogelijk om de applicatie niet significant te doen vertragen. In deze sectie wordt deze impact onderzocht en ge\"evalueerd. \\

Om de performance te meten zijn er twee verschillende applicaties gekozen waar de library in ingebouwd is, namelijk een stopwatch applicatie en een angry birds kloon \cite{AngryBirds}. De stopwatch wordt gebruikt omdat dit een niet CPU intensieve applicatie is en zo de resultaten niet vertekend zijn. De angry birds applicatie is CPU intensiever en vereist een bepaalde framerate, aangezien dit een spel is. Deze applicatie wordt gebruikt om de framerate te meten en de impact van de library op de framerate. \\

Deze evaluaties zijn uitgevoerd op een iPhone 6 \cite{iPhone} met als besturingssysteem: iOS versie 9.3.1. Het toestel was ten allen tijde voorzien met een WiFi verbinding met het internet. Om de resultaten op te halen van het toestel is de ontwikkelaarsomgeving Xcode \cite{Xcode} versie 7.3.1 gebruikt.\\

In de volgende secties worden de metingen uitgelegd en uitgevoerd. Tenslotte wordt hier ook een conclusie uit getrokken. Er gaat gemeten worden hoe lang de gemiddelde uitvoeringstijd bedraagt per methode die gebruikt kan worden van de library. Nadien wordt er gekeken naar de impact van de library op de framerate van de angry birds kloon.\\

\subsection{Call duration}
In deze sectie wordt uitgezocht hoelang het gemiddeld duurt voor elke methode om volledig uitgevoerd te worden. Zo kan er een beeld gevormd worden van hoe sterk de library de applicatie zou vertragen. Om te testen hoe lang een methode gemiddeld duurt om uitgevoerd te worden, is de volgende testopstelling uitgedacht. De geteste methode wordt X aantal keer uitgevoerd om hieruit een gemiddelde tijd te kunnen berekenen. deze X wordt stelselmatig verhoogd om te kijken of de grootte van deze X een invloed heeft op de gemiddelde vertraging van de geteste methode. Deze testopstelling wordt gebruikt om alle beschikbare methodes in de Tracklytics library te evalueren. Het synchronisatieinterval is 60 seconden. \\

In de vorige secties is beschreven dat de developer de keuze gegeven is om de data tijdelijk op de harde schijf op te slaan of dat deze data verloren gaat indien de applicatie volledig afgesloten wordt. Dit heeft een invloed op de snelheid prestaties van de library, omdat er een significant prestatieverschil zit in disk I/O en verwerkingskracht \cite{diskIO}. Om dit prestatieverschil in kaart te brengen is deze evaluatie uitgevoerd eenmaal met het opslaan op disk en eenmaal zonder het opslaan op disk (aangegeven in de resultaten als ZOD).\\

Er is in de Tracklytics library gekozen om de developer de vrijheid te geven om te kiezen of dat de aggregatie van de metingen deels gebeurt op het toestel van de gebruiker of dat deze aggregatie volledig gebeurt in de back end. Omdat er een verschil is in de implementatie van de geaggregeerde meetobjecten en de standaard meetobjecten zijn de testen opgedeeld in de testen voor de standaard meetobjecten en testen voor de geaggregeerde meetobjecten (in het geval van de Gauge, de Meter en de Timer). Zo kan er de juiste conclusies getrokken worden uit de testen die uitgevoerd zijn.\\


De resultaten van het testen van de standaard meetobjecten kunnen gevonden worden in volgende tabellen: \ref{Table:Counter}, \ref{Table:Gauge}, \ref{Table:Histogram}, \ref{Table:Meter}, \ref{Table:Timer}. De resultaten van de testen op de geaggregeerde meetobjecten zijn opgelijst in volgende tabellen: \ref{Table:GaugeAggregate}, \ref{Table:MeterAggregate}, \ref{Table:TimerAggregate}. De conclusies die uit deze resultaten getrokken kunnen worden zijn beschreven in de conclusie sectie \ref{Section:Conclusie}.


\subsection{Invloed op FPS}
De gemiddelde tijd om een methode uit te voeren zegt iets over de vertraging die de library extra invoert in de applicatie. Het zegt echter niets over de impact in de praktijk op een applicatie. Om dit te kunnen testen is er gekozen om de impact van de library op een game te testen. Zoals eerder vermeldt is deze game een angry birds kloon. \\

Om deze test uit te voeren is de volgende test opstelling gebruikt. Er wordt een gemiddelde FPS berekend uit 3000 gemeten waarden. Een FPS waarde wordt elke 0.2 seconde gemeten. Om nu te testen hoe de library een impact op deze FPS heeft, is ervoor gekozen om elke seconde X aantal keer een library call uit te voeren en deze X telkens te verhogen. Er is gekozen om de worst case test uit te voeren gebaseerd op de test van de call duration. In de resultaten is te zien dat het aanmaken van het Counter object met het opslaan op disk het hoogste gemiddelde heeft. Het aanmaken van een Counter wordt gebruikt om de invloed van de library op de FPS te onderzoeken. \\

Een andere factor die invloed kan hebben op de impact die de library heeft op de applicatie is het synchronisatieinterval. Dit interval is een indicatie van hoeveel tijd er tussen twee synchronisatiecycly bevindt. Door dit synchronisatieinterval systematisch te verlagen kan de invloed op de FPS van de applicatie gemeten worden. \\

De resultaten van deze testen kunnen gevonden worden in de volgende tabel \ref{Table:FPS}. De linkerkolom geeft het aantal simultane metingen weer. Het synchronisatieinterval wordt systematisch verlaagd van 60 seconden naar 5 seconden. De conclusies die uit deze resultaten getrokken kunnen worden zijn beschreven in de conclusie sectie \ref{Section:Conclusie}.




\section{Schaalbaarheid}
Het is belangrijk dat de Tracklytics back end een schaalbare back end is. Indien de load op de servers te groot wordt, dan zou dit merkbaar zijn op twee verschillende manieren in de applicatie. De Tracklytics library zou de netwerkinterface constant gebruiken om de metingen te synchroniseren naar de back end. De applicatie waarin de library is ingebouwd deelt deze netwerkinterface met de library. Als gevolg hiervan wordt de applicatie trager, omdat de netwerkinterface constant gebruikt wordt door de library. Dit scenario moet voorkomen worden om de bruikbaarheid van de Tracklytics library optimaal te houden.\\

Een oplossing voor dit probleem is de back end uit te schalen over meerdere servers en servers bij te voegen indien de load op de servers te groot wordt. Om de back end schaalbaar te maken moet zowel de PHP code die op de servers draait schaalbaar zijn, alsook de database die in de back end draait. \\
De PHP code is schaalbaar zolang deze een connectie kan krijgen tot de database, omdat de back end enkel data in de database zet of data eruit haalt. \\

De database gebruikt in de Tracklytics library is een MySQL database. Dit is een implementatie van een Relational database management system (RDBMS). Dit soort database is niet ontwikkeld om schaalbaar te zijn \cite{RDBMS}. Dit soort systeem is ontwikkeld om op \'e\'en server te draaien om de integriteit van de tabel mappings te behouden en de problemen van gedistribueerde verwerking te vermijden. Indien dit systeem dan geschaald moet worden, dan moet er grotere en complexere hardware gekocht worden met meer verwerkingskracht, RAM geheugen en opslag. Een oplossing die hiervoor is gevonden is een master-slave model waarbij de slaves de load van de master kunnen verlagen. \\
Een alternatief hiervoor zou zijn om een NoSQL database te gebruiken. Dit soort database is ontwikkeld om schaalbaar te zijn. Het nadeel hiervan is dat de ACID properties niet gegarandeerd worden, wat ervoor zorgt dat data inconsistent kan worden. Er bestaat ook nog geen standaard taal (zoals SQL voor RDBMS), zodat de overgang van een RDSMS (of van een andere NoSQL) naar een NoSQL database een probleem kan vormen. \\


Een alternief is dat developers van een applicatie de Tracklytics back end hosten op hun eigen servers. Dit is mogelijk, omdat de library open source is. Enkel de connectie met de database moet aangepast worden en de links in de mobiele library moeten vervangen worden door een link naar de server van de developer. Dit geeft als voordeel dat de data bij de developer zelf is en deze hier eventueel andere overzichten van te genereren aangepast aan zijn noden. In dit scenario is de developer niet afhankelijk van een gemeenschappelijke Tracklytics back end en heeft hij de performance van de Tracklytics back end draaiende op zijn servers in eigen handen. Het nadeel aan dit alternatief is dat de developer zelf serverruimte moet voorzien en onderhouden om de back end op te draaien. \\


\section{Developer effort}
Met DevOps willen developers en managers het ontwikkelingsproces van software versnellen. Indien het veel tijd inneemt om de Tracklytics library in een applicatie in te bouwen is het DevOps aspect van deze thesis onvoldoende. Het is ongewenst dat het veel lijnen code inneemt om de libray in de applicatie in te bouwen, omdat dit enerzijds meer tijd inneemt om de code te schrijven en anderzijds de code minder overzichtelijk maakt voor de developers. \\

Om deze eigenschappen na te gaan is ervoor gekozen om volgende zaken na te gaan: hoeveel tijd neemt het in om de library in te bouwen, het aantal lijnen code nodig om deze in te bouwen, wat er gemonitord is en hoeveel metingen er maximaal tegelijkertijd kunnen voorkomen. Met "hoeveel tijd neemt het in om de library in te bouwen" wordt de gehele flow bedoeld: het registreren van de applicatie in de back end, het installeren van de library via CocoaPods en het implementeren van de code in de applicatie. \\

Om deze zaken na te gaan zijn er twee applicaties gebruikt, namelijk de stopwatch applicatie die eerder vermeldt is en een applicatie ontwikkeld in mijn stage. De stopwatch heeft volgende eigenschappen:
\begin{itemize}
\item 250 lijnen code
\item 13u ontwikkelingstijd (inclusief design)
\end{itemize} 

De tweede applicatie heeft volgende eigenschappen:
\begin{itemize}
\item \underline{+} 10000 lijnen code
\item 4 maanden ontwikkelingstijd
\end{itemize}

\subsection{Resultaten}
In deze sectie worden de resultaten besproken van de evaluatie van de developer effort. In de volgende sectie worden deze resultaten besproken \ref{Section:Conclusie}.
\subsubsection{StopWatch applicatie}
\begin{itemize}
\item ​​Duratie: Minder dan 8 minuten
\item \#Lijnen code:\begin{itemize}
\item 9 meetobjecten in totaal
\item ​gemiddeld 1.3 per meetobject(zonder imports)
\item totaal 9 lijnen
\end{itemize}

\item ​​​Tracken van: alle buttons, tijd wanneer op stop gedrukt, aantal rondes alvorens reset, tijd om data van een server te halen.

\item Maximaal 2 metingen tegelijk (Counter \& Gauge)
\end{itemize}




\subsubsection{Andere applicatie}
\begin{itemize}
\item ​Duratie: \underline{+} anderhalf uur
\item \#​Lijnen code: \begin{itemize}
\item 72 meetobjecten in totaal
\item ​​gemiddeld 1.4 per meetobject (zonder imports)
\item totaal \underline{+} 100 lijnen code
\end{itemize}

\item ​Tracken van: Alle buttons waarop geklikt, alle screen visits, \#zoekresultaten, duratie van complexe code, duratie content van het netwerk halen
\item Maximaal \underline{+} 10 metingen tegelijk (Counters, Timers en Gauges)
\end{itemize}


Het is nuttig om dit soort informatie te meten, omdat met behulp van dit soort metingen kan ondervonden worden hoe de gebruiker de applicatie gebruikt. Er kan zo een overzicht gevormd worden welke knoppen hoe vaak gebruikt worden en welke schermen dat de gebruiker het vaakst bezoekt. Anderzijds is het monitoren van de duratie van complexe code en van het ophalen van informatie van het netwerk een manier om de bottleneck te kunnen ontdekken in de applicatie. 

\section{Conclusie}\label{Section:Conclusie}








\begin{sidewaystable}
\centering
\begin{tabular}{|l|l|l|l|l|l|l|}
\hline
\# metingen   & Aanmaken     & Inc         & Dec         & Aanmaken ZOD & Inc ZOD     & Dec ZOD     \\ \hline
1         & 0,003168987  & 0,000516049 & 0,000513007 & 4,41015E-06  & 5,87324E-07 & 5,41994E-07 \\ \hline
10        & 0,001451302  & 0,000512397 & 0,000505489 & 1,68934E-06  & 5,30546E-07 & 5,29546E-07 \\ \hline
100       & 0,001045819  & 0,000505216 & 0,000506907 & 1,18792E-06  & 5,50558E-07 & 5,7864E-07  \\ \hline
1000      & 0,001175395  & 0,000522625 & 0,000505655 & 2,7231E-06   & 5,35803E-07 & 5,27143E-07 \\ \hline
10000     & 0,0017926232 & 0,000545799 & 0,000527276 & 2,05297E-06  & 5,54156E-07 & 5,10363E-07 \\ \hline
50000     & 0,003394888  & 0,00053188  & 0,000539547 & 1,7636E-06   & 5,45784E-07 & 5,8755E-07  \\ \hline
100000    & 0,005672893  & 0,000534656 & 0,0005382   & 2,30629E-06  & 5,26104E-07 & 4,93241E-07 \\ \hline
200000    & 0,010294359  & 0,000530225 & 0,000537563 & 1,66409E-06  & 4,97316E-07 & 5,43947E-07 \\ \hline
300000    & 0,014826151  & 0,000553307 & 0,000544871 & 1,76572E-06  & 5,45442E-07 & 5,08001E-07 \\ \hline
Gemiddeld & 0,004758046  & 0,000528017 & 0,000524279 & 2,17369E-06  & 5,41448E-07 & 5,35603E-07 \\ \hline
\end{tabular}
\caption{Resultaten call tijd Counter (in seconden)}
\label{Table:Counter}
\end{sidewaystable}

\begin{table}[]
\centering
\begin{tabular}{|l|l|l|l|l|}
\hline
\# metingen   & Aanmaken    & Stop        & Aanmaken ZOD & Stop ZOD    \\ \hline
1         & 0,000342011 & 0,000580966 & 4,09013E-06  & 2,89679E-05 \\ \hline
10        & 0,000284296 & 0,000492996 & 1,91528E-06  & 5,6982E-06  \\ \hline
100       & 0,000298141 & 0,00049423  & 2,07685E-06  & 2,2608E-06  \\ \hline
1000      & 0,000297281 & 0,000522041 & 2,03055E-06  & 2,18725E-06 \\ \hline
10000     & 0,000740984 & 0,000527625 & 1,88986E-06  & 1,26034E-06 \\ \hline
50000     & 0,002660037 & 0,000536449 & 2,03353E-06  & 1,17797E-06 \\ \hline
100000    & 0,00515636  & 0,000544189 & 1,83201E-06  & 1,17261E-06 \\ \hline
200000    & 0,009800656 & 0,000575258 & 1,86301E-06  & 1,16086E-06 \\ \hline
300000    & 0,014302578 & 0,000522853 & 1,75735E-06  & 1,11202E-06 \\ \hline
Gemiddeld & 0,003764705 & 0,000532956 & 2,1654E-06   & 4,99977E-06 \\ \hline
\end{tabular}
\caption{Resultaten call tijd Timer (in seconden)}
\label{Table:Timer}
\end{table}

\begin{table}[]
\centering
\begin{tabular}{|l|l|l|l|l|}
\hline
\# metingen & Aanmaken    & Stop        & Aanmaken ZOD & Stop        \\ \hline
1           & 0,002120972 & 0,002666599 & 4,36699E-05  & 4,00181E-06 \\ \hline
10          & 0,000227898 & 0,002475894 & 3,0541E-05   & 1,69697E-06 \\ \hline
100         & 2,12032E-05 & 0,000657661 & 2,74795E-05  & 1,77817E-06 \\ \hline
1000        & 3,52955E-06 & 0,000614075 & 3,24672E-06  & 1,51634E-06 \\ \hline
10000       & 1,26566E-06 & 0,000587031 & 1,25878E-06  & 1,43751E-06 \\ \hline
50000       & 1,0892E-06  & 0,000584363 & 1,12291E-06  & 1,36487E-06 \\ \hline
100000      & 1,02262E-06 & 0,000574846 & 1,04058E-06  & 1,45262E-06 \\ \hline
200000      & 1,03471E-06 & 0,000565405 & 1,02583E-06  & 1,36E-06    \\ \hline
300000      & 1,0063E-06  & 0,000574488 & 1,01804E-06  & 1,4087E-06  \\ \hline
Gemiddeld   & 0,000264336 & 0,001033374 & 1,2267E-05   & 1,77967E-06 \\ \hline
\end{tabular}
\caption{Resultaten call tijd Timer met Aggregatie (in seconden)}
\label{Table:TimerAggregate}
\end{table}

\begin{table}[]
\centering
\begin{tabular}{|l|l|l|}
\hline
\# metingen & Aanmaken    & Aanmaken ZOD \\ \hline
1           & 0,002752006 & 4,21825E-06  \\ \hline
10          & 0,001020319 & 1,94199E-06  \\ \hline
100         & 0,001067632 & 2,11683E-06  \\ \hline
1000        & 0,001118987 & 2,89034E-06  \\ \hline
10000       & 0,001677553 & 2,00928E-06  \\ \hline
50000       & 0,003420756 & 1,86953E-06  \\ \hline
100000      & 0,005801659 & 1,86129E-06  \\ \hline
200000      & 0,010283894 & 1,85536E-06  \\ \hline
300000      & 0,014828387 & 1,8752E-06   \\ \hline
Gemiddeld   & 0,004663466 & 2,29312E-06  \\ \hline
\end{tabular}
\caption{Resultaten call tijd Histogram (in seconden)}
\label{Table:Histogram}
\end{table}


\begin{table}[]
\centering
\begin{tabular}{|l|l|l|}
\hline
\# metingen & addEntry    & addEntry ZOD \\ \hline
1           & 0,002198004 & 3,99419E-06  \\ \hline
10          & 0,001131004 & 2,17136E-06  \\ \hline
100         & 0,001070941 & 2,1365E-06   \\ \hline
1000        & 0,001186648 & 2,36198E-06  \\ \hline
10000       & 0,001511716 & 2,28242E-06  \\ \hline
50000       & 0,003305941 & 2,04427E-06  \\ \hline
100000      & 0,005528791 & 2,23297E-06  \\ \hline
200000      & 0,010175614 & 1,88071E-06  \\ \hline
300000      & 0,015041323 & 1,89719E-06  \\ \hline
Gemiddeld   & 0,00457222  & 2,33351E-06  \\ \hline
\end{tabular}
\caption{Resultaten call tijd Meter (in seconden)}
\label{Table:Meter}
\end{table}

\begin{table}[]
\centering
\begin{tabular}{|l|l|l|}
\hline
\# metingen & addEntry    & addEntry ZOD \\ \hline
1           & 0,025438964 & 4,48893E-05  \\ \hline
10          & 0,003257507 & 2,90151E-05  \\ \hline
100         & 0,000962322 & 6,39166E-06  \\ \hline
1000        & 0,000811738 & 1,56662E-06  \\ \hline
10000       & 0,000793704 & 1,18229E-06  \\ \hline
50000       & 0,000813867 & 1,15234E-06  \\ \hline
100000      & 0,000769992 & 1,0986E-06   \\ \hline
200000      & 0,000618082 & 1,15231E-06  \\ \hline
300000      & 0,000561398 & 1,10977E-06  \\ \hline
Gemiddeld   & 0,003780842 & 9,72867E-06  \\ \hline
\end{tabular}
\caption{Resultaten call tijd Meter met Aggregatie (in seconden)}
\label{Table:MeterAggregate}
\end{table}


\begin{table}[]
\centering
\begin{tabular}{|l|l|l|}
\hline
\# metingen & Aanmaken    & Aanmaken ZOD \\ \hline
1           & 0,002469957 & 4,12756E-06  \\ \hline
10          & 0,001076302 & 1,95558E-06  \\ \hline
100         & 0,001117599 & 2,09727E-06  \\ \hline
1000        & 0,001157752 & 2,47002E-06  \\ \hline
10000       & 0,001552656 & 2,00483E-06  \\ \hline
50000       & 0,003327206 & 1,81246E-06  \\ \hline
100000      & 0,005833407 & 1,91939E-06  \\ \hline
200000      & 0,010283247 & 1,82617E-06  \\ \hline
300000      & 0,015243511 & 1,90657E-06  \\ \hline
Gemiddeld   & 0,004673515 & 2,23554E-06  \\ \hline
\end{tabular}
\caption{Resultaten call tijd Gauge (in seconden)}
\label{Table:Gauge}
\end{table}


\begin{table}[]
\centering
\begin{tabular}{|l|l|l|}
\hline
\# metingen & Aanmaken    & Aanmaken ZOD \\ \hline
1           & 0,018877983 & 0,002133965  \\ \hline
10          & 0,002148187 & 0,0004381    \\ \hline
100         & 0,00066519  & 0,000245582  \\ \hline
1000        & 0,000668403 & 0,000252874  \\ \hline
10000       & 0,001156017 & 0,000333561  \\ \hline
50000       & 0,002451678 & 0,001636235  \\ \hline
100000      & 0,004218748 & 0,003317311  \\ \hline
200000      & 0,007491983 & 0,006745731  \\ \hline
300000      & 0,011473117 & 0,01040852   \\ \hline
Gemiddeld   & 0,005461256 & 0,002834653  \\ \hline
\end{tabular}
\caption{Resultaten call tijd Gauge met Aggregatie (in seconden)}
\label{Table:GaugeAggregate}
\end{table}


\begin{table}[]
\centering
\begin{tabular}{|l|l|l|l|l|l|}
\hline
\#metingen/sec & 60 sec sync & 30 sec sync & 15 sec sync & 10 sec syn  & 5 sec sync \\ \hline
0              & 31,5632141  & 31,6737281  & 31,5281931  & 31,5672819  & 31,0382918 \\ \hline
1              & 31,5482895  & 31,5798341  & 31,5577123  & 31,5036609  & 31,2145193 \\ \hline
10             & 31,4920183  & 31,5852394  & 31,1482867  & 31,2389401  & 31,2015632 \\ \hline
20             & 30,9340218  & 30,9576821  & 30,8974831  & 30,91473821 & 30,7183928 \\ \hline
50             & 29,1320184  & 29,3209473  & 29,3382103  & 29,1371842  & 27,0873822 \\ \hline
100            & 20,8572947  &             &             &             &            \\ \hline
1000           & 9,3930281   &             &             &             &            \\ \hline
\end{tabular}
\caption{Resultaten metingen van het aanmaken van een Counter (in FPS)}
\label{Table:FPS}
\end{table}


%%% Local Variables: 
%%% mode: latex
%%% TeX-master: "masterproef"
%%% End: 
