\chapter{Evaluatie}
In deze sectie wordt de ontwikkelde library ge\"evalueerd. Allereerst wordt er gekeken naar wat de impact van de library op de performance van de applicatie is. Verder in dit hoofdstuk wordt de schaalbaarheid van de library bekeken. Ten slotte wordt er nog gekeken hoeveel effort er van de developer nodig is om de library te integreren in een applicatie.



\section{Performance}
In het verloop van deze thesis is al enkele keren aan bod gekomen dat de prestaties van een applicatie \'e\'en van de belangrijkste eigenschappen van een applicatie is. De impact op de performance van de library op de performance moet aanvaardbaar zijn en zo klein mogelijk om de applicatie niet significant te doen vertragen. In deze sectie wordt deze impact onderzocht en ge\"evalueerd. \\

Om de performance te meten zijn er twee verschillende applicaties gekozen, namelijk een stopwatch applicatie en een angry birds kloon \cite{AngryBirds}. De stopwatch wordt gebruikt omdat dit een niet CPU intensieve applicatie is en zo de resultaten niet vertekend zijn. De angry birds applicatie is CPU intensiever en vereist een bepaalde framerate, aangezien dit een spel is. Deze applicatie wordt gebruikt om de framerate te meten en de impact van de applicatie hierop. \\

In de volgende secties worden de metingen uitgelegd en uitgevoerd. Tenslotte wordt hier ook een conclusie uit getrokken. Er gaat gemeten hoe lang het gemiddeld duurt per methode die gebruikt kan worden van de library. Dit wordt gemeten langs twee aspecten: de data wordt wel tijdelijk op harde schijf opgeslagen en de data wordt niet tijdelijk op harde schijf opgeslagen. Nadien wordt er gekeken naar de impact van de library op de framerate van de angry birds kloon. Dit laatste combineren we met het wel of niet aggregeren van de data op het toestel zelf. \\

\subsection{Call duration}
In deze sectie wordt uitgezocht hoelang het gemiddeld duurt voor elke methode om volledig uitgevoerd te worden. Zo kan er een beeld gevormd worden van hoe sterk de library de applicatie zou vertragen.




\section{Schaalbaarheid}


\section{Developer effort}


%%% Local Variables: 
%%% mode: latex
%%% TeX-master: "masterproef"
%%% End: 
