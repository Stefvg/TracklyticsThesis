\chapter{Besluit}
\label{besluit}


In deze thesis werd er op zoek gegaan naar een oplossing om het ontwikkelen van mobiele applicaties en de concepten van DevOps te combineren. Er werd gekozen om te kijken naar het monitoring aspect van DevOps. Er zijn enkele oplossingen voorgesteld om mobiele applicaties te monitoren, namelijk: built-in OS support, een dedicated test environment en een monitoring library. Uit deze voorstellen werd \'e\'en voorstel gekozen, namelijk een library om in een mobiele applicatie in te bouwen die de applicatie monitort. In Hoofdstuk \ref{doelstelling} wordt beschreven welke specificaties de library minimaal moet hebben om een succesvolle library te kunnen zijn. De vormgeving en architectuur van de monitoring library wordt voorgesteld in het hoofdstuk architectuur samen met enkele uitbreidingen op de library. Deze architectuur wordt ge\"implementeerd in een library die ontwikkeld is voor iOS, het besturingssysteem voor mobiele toestellen van Apple. Deze implementatie werd ten slotte ge\"evalueerd om te kijken of deze kan werken in de realiteit. \\


In Hoofdstuk \ref{doelstelling} zijn een aantal doelstellingen opgesomd voor de ontwikkelde library. Deze werden opgedeeld in volgende categorie\"en: 
\begin{itemize}
\item performance impact
\item schaalbaarheid
\item bruikbaarheid
\item beschikbaarheid
\end{itemize}

\paragraph{De performance impact} wordt gezien als het belangrijkste aspect in het ontwikkelen van een monitoring library. Deze impact werd onderzocht en getest in het hoofdstuk evaluatie \ref{evaluatie}. Uit deze evaluatie kan afgeleid worden dat de impact van de library op een mobiele applicatie klein genoeg is om de applicatie niet significant te vertragen. \\


\paragraph{De schaalbaarheid} van de Tracklytics library werd besproken in Hoofdstuk \ref{evaluatie}. De back end is het deel van de library dat schaalbaar moet zijn om de requests die van de mobiele library komen te verwerken. De bottleneck in de back end is de relationele database. Deze moet zo goed mogelijk uitgeschaald worden om aan de requests te kunnen voldoen. Een alternatief is om de relationele database te veranderen naar een NoSQL of andere schaalbare database. Het is ook mogelijk voor de developer om de back end op de eigen servers te draaien om zo alle data in eigen bezit te hebben en niet afhankelijk te zijn van de Tracklytics infrastructuur.\\


\paragraph{De bruikbaarheid} van de library wordt gedefini\"eerd als hoeveel moeite een developer nodig heeft om de library in te bouwen in de mobiele applicatie. In Hoofdstuk \ref{evaluatie} werd nagegaan hoeveel moeite een developer nodig heeft om de library in de applicatie in te bouwen. In deze evaluatie werden de volgende zaken opgenomen: de totale tijd om de library in te bouwen, het aantal lijnen code en wat er gemonitord wordt door de library. Uit de gegevens die hieruit zijn gekomen kan geconcludeerd worden dat de developer effort minimaal is.

\paragraph{Om de beschikbaarheid} van de data maximaal te houden moet elke meting naar de back end verstuurd worden. De Tracklytics library laat de keuze of de data tijdelijk op de harde schijf van het toestel opgeslagen moet worden over aan de developer zelf. De developer moet deze keuze aangeven in het dashboard. \\

De voorgaande paragrafen tonen aan dat de doelstellingen vooropgesteld in deze thesis voldaan zijn in de implementatie van de library. Naast deze doelstellingen werd in Hoofdstuk \ref{architectuur} nog enkele vereisten gegeven waaraan de library moet voldoen om de developer te helpen bij het ontwikkelen van een mobiele applicatie, namelijk: 
\begin{itemize}
\item op welke buttons/switches/entry in een tabel gebruikers drukken
\item welke schermen de gebruikers bezoeken
\item het gemiddeld aantal zoekresultaten per zoekopdracht
\item het gemiddelde of de verdeling van het getal dat een gebruiker in een bepaald veld invoert
\item hoe lang de applicatie gemiddeld gebruikt wordt
\item hoe lang een stuk code over het uitvoeren ervan doet om na te gaan of deze code niet te traag is.
\item de tijd die een request over het internet nodig heeft om te voltooien om te kijken of er hier een vertraging opgelopen wordt.
\item het gemiddeld aantal entries in een array of een NSDictionary (een Map in Java). Indien er ge\"itereerd wordt over deze datastructuren kan een groot aantal entries ervoor zorgen dat dat stuk code de applicatie vertraagt.
\item het gemiddeld aantal keer dat een bepaalde methode uitgevoerd wordt om te kijken
\end{itemize}

Met behulp van de Tracklytics library kan aan deze vereisten voldaan worden. De eerste twee vereisten kunnen ingelost worden door counters in te bouwen in de applicatie. De tweede, derde, voorlaatste en laatste vereiste kunnen opgelost worden door ofwel een gauge te gebruiken ofwel een histogram te gebruiken. Aan de vierde, de vijfde en de zesde vereiste kan voldaan worden door een timer in te bouwen in de applicatie.\\

Een monitoring library heeft zijn voor- en nadelen. De gebruikers kunnen de applicatie anders gebruiken dan dat deze getest wordt in een testopstelling. Een monitoring library kan deze informatie ophalen, omdat deze in de applicatie, die in productie is, ingebouwd is. Het nadeel aan een monitoring library is dat deze een extra performance impact heeft op de applicatie, maar zoals besproken in Hoofdstuk \ref{evaluatie} is deze impact klein genoeg. Het dashboard zou nog wat meer functionaliteit kunnen hebben door nog gedetailleerdere overzichten te geven of extra mogelijkheden te geven tot het verkleinen van de dataset. \\

Indien ik meer tijd zou hebben gehad in het ontwikkelen van deze thesis zou ik graag nog wat uitbreidingen toegevoegd hebben aan de monitoring library. Als eerste zou ik graag het tracken van features hebben toegevoegd aan de applicatie. Features zijn een samenhang van componenten die een functie uitvoeren. Naar dit soort tracking van applicaties wordt momenteel veel onderzoek gedaan (bijvoorbeeld \cite{fischer2003analyzing}) en ik zou dit onderzoek willen combineren met mobiele applicaties. \\
Daarnaast zou ik graag AB testing in de applicatie inbouwen. AB testing is het concept dat een selecte groep van de gebruikers een nieuwere versie te zien krijgen dan een andere groep. Zo kan de uitrol van een nieuwe versie geleidelijk aan gebeuren en indien er een bug zit in de software is deze enkel zichtbaar voor een klein deel van de gebruikers. In mobiele applicaties is dit een uitdaging, omdat dynamisch code laden niet mogelijk is. Ik zou willen onderzoeken wat de mogelijkheden hierin zijn en welke alternatieven er zijn om toch aan AB testing te kunnen doen en deze dan te combineren met de monitoring library.\\
Ten slotte zou ik willen onderzoeken of het mogelijk is om een plugin te bouwen voor Xcode, de ontwikkelingsomgeving voor het iOS besturingssysteem. Deze plugin zou de developer kunnen helpen met het ontwikkelen van een applicatie door bijvoorbeeld automatisch aan te geven welke stukken code het traagste zijn op basis van de metingen door de Tracklytics library. \\
 
%%% Local Variables: 
%%% mode: latex
%%% TeX-master: "masterproef"
%%% End: 
