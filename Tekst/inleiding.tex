\chapter{Inleiding}
\label{inleiding}


DevOps is een concept, gelanceerd in 2008 \cite{DevOpsWiki}, om developers en IT operation professionals beter de laten samenwerken. Dit zorgt ervoor dat, op moment van schrijven, het concept nog relatief nieuw is en er nog veel onderzoek gedaan kan worden naar gebieden waar DevOps kan helpen om de kwaliteit van software te verbeteren en hoe software sneller op de markt kan gebracht worden. \'E\'en van die gebieden is het ontwikkelen van mobiele applicaties. Hoewel dit aspect van software ontwikkeling steeds populairder wordt, is het onderzoek hiernaar nog steeds minimaal. De uitdaging is om de DevOps concepten te combineren met het ontwikkelen van mobiele applicaties. \\

Deze thesis gaat deze uitdaging aan en probeert DevOps te integreren in het ontwikkelen van mobiele applicaties. Er wordt gefocust op \'e\'en aspect van de DevOps concepten, namelijk het monitoren van mobiele applicaties. Dit is de rode draad doorheen deze thesis. Er wordt gezocht naar mogelijkheden om deze uitdaging te volbrengen. \\

Om dit onderzoek verder in te leiden wordt de context uitgelegd waarin deze thesis zich bevindt. In het context hoofdstuk worden de DevOps concepten uitgelegd om een beeld te schetsen waar het onderzoek zich op focust. Later wordt er besproken waarom het relevant is dat mobiele applicaties gemonitord worden. Deze stellingen worden ondersteund door het geven van enkele development scenario's die in de realiteit kunnen voorkomen. Er worden enkele voorstellen tot oplossingen voor het probleem beschreven met hun voor- en nadelen.\\

Uit de voorgestelde oplossingen wordt in deze thesis \'e\'en oplossing gekozen, namelijk het monitoren van mobiele applicaties door een library in te bouwen. In het hoofdstuk doelstellingen wordt er beschreven welke doelstellingen deze library moet hebben om een succesvolle library te vormen. De doelstellingen worden opgesteld aan de hand van twee aspecten, namelijk de impact op de performance van de applicatie en de hoeveelheid moeite een developer in het inbouwen van de library moet steken. Nadien wordt er besproken hoe data kan worden weergegeven om de developer te kunnen helpen bij het ontwikkelen van een mobiele applicatie.\\

De doelstellingen worden omgevormd in een architectuur van de library die ontwikkeld wordt in deze thesis. Aan de hand van diagrammen wordt aangegeven hoe de componenten van de library samenwerken om de applicatie te monitoren. Ten slotte worden uitbreidingen besproken die de library verrijken met extra functionaliteit om developers meer vrijheid te geven in het monitoren van de applicatie. \\

Deze architectuur gecombineerd met enkele uitbreidingen is ge\"implementeerd in een functionele library voor iOS, het mobiele besturingssysteem van Apple. In het hoofdstuk over de implemantatie wordt uitgelegd hoe de structuur van de library in elkaar zit en welke technologie\"en gebruikt zijn om de library te schrijven. Daarnaast wordt de implementatie van de back end uitgelegd en er wordt besproken hoe de mobiele library zich connecteert met de back end. Om de data weer te geven is een dashboard ontwikkeld waarop deze data grafisch wordt weergegeven. Welke technologie\"en gebruikt worden en welke keuzes gemaakt zijn om dit dashboard te ontwikkelen en de data data weer te geven worden in dit hoofdstuk uitgelegd. Ten slotte worden de openstaande uitdagingen bekeken die als uitbreiding kunnen worden ge\"implementeerd in deze library.\\


Het is belangrijk dat developers weten hoe ze deze library kunnen inbouwen in de applicatie die ze wensen te monitoren. Om developers hierbij te helpen wordt er een tutorial gegeven die stap voor stap aangeeft hoe de library ingebouwd kan worden in de applicatie. Er wordt uitgelegd hoe elk type meetobject gebruikt kan worden en welke parameters nodig zijn om de library te doen functioneren zoals het hoort. \\

Om te kijken of de ontwikkelde library voldoet aan de doelstellingen die voorop werden gesteld, hebben we de library ge\"evalueerd. Er wordt eerst gekeken naar de impact die de library heeft op de performance van de applicatie door te kijken naar hoe lang het duurt om de methodes uit te voeren enerzijds en anderzijds te kijken naar welke invloed de library heeft op de framerate van een applicatie. Daarnaast wordt er gekeken naar hoe schaalbaar de back end van de library is. Naast de performance van de applicatie wordt er gekeken hoeveel moeite een developer nodig heeft om de library in te bouwen in de applicatie. Ten slotte wordt er uit deze resultaten een conclusie getrokken omtrent de library. \\

Om deze thesis af te sluiten wordt er een besluit getrokken uit alle conclusies en stellingen die gemaakt zijn in de voorgaande hoofdstukken. \\


%%% Local Variables: 
%%% mode: latex
%%% TeX-master: "masterproef"
%%% End: 
