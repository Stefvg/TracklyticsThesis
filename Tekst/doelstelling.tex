\chapter{Doelstelling} \label{doelstelling}
Het onderzoek van deze thesis baseert zich op twee invalshoeken, enerzijds hoe de developers en eigenaars van een mobiele applicatie geholpen kunnen worden bij het ontwikkelen en verbeteren van die applicatie en anderzijds wat de trade-off is tussen enerzijds hoeveel de applicatie gemonitord wordt en anderzijds de impact op de performance van de applicatie, de developer effort, etc. \\

De eerste invalshoek gaat over hoe we developers kunnen helpen bij het ontwikkelen van een mobiele applicatie door de developers problemen te laten identificeren (zoals trage code sequenties) en de eigenaars te laten ontdekken welke componenten van de applicatie het meest gebruikt worden en welke het minste. Met deze informatie weten de developers waar ze verbeteringen kunnen aanbrengen en weten eigenaren in welke functies ze wel of niet moeten investeren. Langs de andere kant is het belangrijk om te weten wat de impact is van het verkrijgen van de informatie op de prestaties van de applicatie en hoeveel moeite en tijd er nodig is van de developer om deze informatie te kunnen vergaren.\\


Om dit onderzoek te doen slagen wordt er een library gebouwd die developers in hun applicaties kunnen inbouwen om hun applicatie te monitoren. In de volgende sectie wordt uitgelegd wat het doel is van deze library. In het verdere verloop van deze thesis wordt uitgelegd hoe de architectuur achter de gebouwde library in elkaar zit en waarom de keuzes zijn gemaakt voor die structuur. Nadat deze architectuur beschreven is, wordt er een uitleg gegeven over de specifieke implementatie van de library. Ten slotte wordt de library ge\"evalueerd om te kijken of de kwaliteit van de library voldoende is om gebruikt te worden in de applicaties. 

\section{Library}
Het doel van de library is om developers deze library te laten inbouwen in hun applicatie en met behulp van de methodes aangeboden door deze library de applicatie te kunnen monitoren. Deze methodes moeten voldoende zijn om alle aspecten van de applicatie te kunnen monitoren. \\

Om een kwaliteitsvolle library te bouwen zijn er een aantal aspecten die belangrijk zijn om in het achterhoofd te houden:
\begin{itemize}
\item performance impact
\item schaalbaarheid
\item bruikbaarheid
\item beschikbaarheid
\end{itemize}

Deze aspecten worden besproken in het komende deel van dit hoofdstuk. 

\subsection{Performance impact}
Het is van cruciaal belang dat een applicatie niet traag is en responsief is naar de gebruiker toe. De verwerkingskracht (CPU) van een mobiel apparaat is relatief beperkt. Deze twee eigenschappen gecombineerd zorgt ervoor dat de library een zo beperkt mogelijke impact mag hebben op deze verwerkingskracht. Indien deze library relatief veel CPU tijd inneemt, dan blijft er minder CPU tijd over voor de applicatie, wat ervoor zorgt dat de applicatie trager wordt indien deze CPU-intensief is. Het is dus van cruciaal belang dat de library geoptimaliseerd wordt om zo weinig mogelijk CPU tijd te verbruiken. Een bijkomend voordeel dat verschijnt indien de library zo minimaal mogelijk CPU tijd verbruikt is dat de impact van de applicatie op het batterijverbruik verminderd wordt.  \\

Naast het CPU verbruik is het ook van belang dat het netwerkgebruik zo effici\"ent mogelijk gebruikt wordt. In een mobiel apparaat wordt de netwerkinterface in slaapstand gehouden tot deze gebruikt moet worden en op dat moment wordt deze interface geactiveerd. Dit wordt gedaan omdat de netwerkinterface relatief veel energie verbruikt indien deze aanstaat. Indien de netwerkinterface zo effici\"ent mogelijk gebruikt wordt kan deze vaker in slaapstand gezet worden en is de impact van de library op het batterijgebruik minder dan indien deze ineffici\"ent gebruikt wordt. Deze effici\"entie zorgt er ook voor dat de applicatie de netwerkinterface zo vaak mogelijk kan gebruiken en hier zo weinig mogelijk vertraging opgelopen wordt als de applicatie het netwerk wil gebruiken. \\

Niet enkel de effici\"entie van het gebruik van de netwerkinterface is van belang, maar ook de snelheid waarmee de back end reageert op de request en deze een resultaat kan terug geven. Indien het relatief lang duurt eer de back end het resultaat terug geeft, is de optimalisatie van het gebruik van de netwerkinterface voor niets geweest. De snelheid van de back end is dus ook een belangrijk aspect. \\

De impact op de performance van de applicatie is het belangrijkste aspect in het bouwen van een monitoring library voor mobiele applicaties. Indien deze impact relatief groot is, dan is het voor developers onmogelijk om deze library in de applicatie in te bouwen en nog steeds een degelijke responsiviteit en snelheid van deze applicatie te behalen. 

\subsection{Schaalbaarheid}
Zoals hierboven vermeldt, is de snelheid van de back end een belangrijk aspect inzake performance van de netwerkinterface. \'E\'en aspect van deze snelheid is de implementatie van deze back end. De implementatie moet geoptimaliseerd worden zodat deze zo snel mogelijk een resultaat kan teruggeven aan de library. Het andere aspect in de snelheid van de back end is de load die de servers aankunnen om al de requests te verwerken. De mate van schaalbaarheid is hier een belangrijke factor in. \\

Een applicatie wordt simultaan gebruikt door gemiddeld \textit{N} gebruikers waarin de library aanwezig is. Deze applicatie verstuurt per keer gemiddeld \textit{X} metingen. De library wordt gebruikt door \textit{M} verschillende applicaties. De applicaties zijn geconfigureerd met een gemiddeld synchronisatie-interval \textit{T}. Dit geeft dat het gemiddeld aantal requests per tijdseenheid neerkomt op: $RpT=\frac{M*N*X}{T}$. Hoe hoger dit getal, hoe hoger het aantal requests de servers moeten beantwoorden en hoe meer load er per server komt. Elke server \textit{S} heeft een limiet kwa aantal requests die simultaan kunnen beantwoord worden \textit{R}. De volgende vergelijking geeft de relatie aan tussen het gemiddeld aantal requests en het aantal requests dat beantwoord kan worden: $S*R > RpT$. Dit wil zeggen dat er (altijd) meer requests moeten kunnen beantwoord worden dan dat er gemiddeld per tijdseenheid aan komen.\\



\subsection{Bruikbaarheid}
Het primaire doel van een library is om de taken uit te voeren die worden gegeven aan de library. Naast dit primaire doel is het ook de bedoeling dat developers de library kunnen gebruiken zonder veel effort. De bruikbaarheid van de library moet zo hoog mogelijk zijn zodat developers zonder veel tijd en moeite deze kunnen integreren in een applicatie. De volgende punten zijn hierin belangrijk: 

\begin{itemize}
\item concreet willen we maximaal vermijden dat de library de developer hindert bij het ontwikkelen van de applicatie;

\item concreet hebben we als doelstelling het vereiste aantal lijnen code en configuratie zo klein mogelijk te houden;

\item concreet willen we dat de developer een zo minimaal mogelijke tijd moet spenderen bij het implementeren van de library.
\end{itemize}




\subsection{Beschikbaarheid}
Met beschikbaarheid wordt bedoeld dat alle gecollecteerde data opgeslagen wordt op de servers. De data die gecollecteerd wordt is waardevol. Het is de bedoeling dat er zo'n goed mogelijk beeld geschetst wordt voor de developer. Indien een deel van de data niet opgeslagen wordt in de database kan het zijn dat mogelijke uitzonderlijke data niet opgeslagen wordt, terwijl deze uitzonderlijke data juist de belangrijkste is. Er moet dus voor gezorgd worden dat 100\% van de gecollecteerde data gesynchroniseerd wordt naar de server om een zo goed mogelijk beeld te vormen voor de developers en de eigenaars van de applicatie.\\


Deze aspecten moeten in rekening genomen worden bij het ontwerpen en ontwikkelen van een mobiele monitoring library. De belangrijkste aspecten zijn de performance aspecten, omdat in mobiele applicaties de verwerkingskracht en batterij de bottlenecks zijn. 


\section{Data Visualisatie}
Data visualisatie is een belangrijk aspect in de library, omdat de developer uit deze visualisaties conclusies moet kunnen trekken over de prestaties en het gebruik van de applicatie. De verzamelde data van de gebruikers is samengenomen in \'e\'en centraal punt. Het doel is om deze data samen te voegen en hieruit de nodige informatie te halen en weer te geven aan de developer. Het is dus de zaak om uit te zoeken voor elk soort meting welke informatie belangrijk is. Deze informatie moet worden gevisualiseerd om de developer een compacter overzicht te geven van de resultaten van de metingen. Door de data te groeperen op eigenschappen kan de developer dieper ingaan op de details van de metingen. Indien er niet aan data visualisatie gedaan zou worden, zou de developer zelf de verschillende parameters moeten berekenen uit de ruwe data. \\

Het uiteindelijke doel is dus om per type meting de parameters te gaan identificeren die nuttig kunnen zijn voor de developer om een conclusie uit dat type meting te kunnen trekken. Daarnaast is het de bedoeling om een software pakket te maken die deze parameters gaat weergeven in een overzicht voor de developer.

%%% Local Variables: 
%%% mode: latex
%%% TeX-master: "masterproef"
%%% End: 
