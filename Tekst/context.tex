\chapter{Context}

\section{Relevantie monitoren mobiele applicaties}
Op het moment van schrijven is het 2016; het jaar waarin ongeveer \'e\'en op de vijf mensen een smartphone gebruikt in het dagelijkse leven. Een smartphone bevat meerdere mobiele applicaties, ontwikkeld door developers wereldwijd. In de App Store van Apple staan ongeveer 1.5 miljoen verschillende applicaties. Een karakteristiek van de mobiele app wereld is dat er voor een bepaalde app vaak een of meerdere alternatieven bestaan. Dit wil zeggen dat de eigenaars van de app zich moeten proberen onderscheiden van de andere apps. Dat kan op meerdere manieren: door een praktisch design, door een snellere app te hebben, etc. Door tijd en/of geld te investeren in het verbeteren van de applicatie kan de applicatie de meest gebruikte applicatie worden en blijven in zijn categorie. \\
%cite http://www.statista.com/statistics/330695/number-of-smartphone-users-worldwide/

Met het design wordt de structuur van de gebruikersinterface van de applicatie bedoeld. Dit design bepaalt waar welke UI elementen komen te staan. Er wordt hier niet bedoeld op de esthetiek van de gebruikersinterface.
Het design wordt meestal op voorhand vastgelegd alvorens de applicatie ontwikkeld wordt, zodat de developer dit kan gebruiken bij het implementeren. Er kan niet objectief gezegd worden wat een goed of een slecht design is door een ontwikkelaar omdat dit een persoonlijke mening is. Het design kan enkel \'echt beoordeeld worden door de gebruikers van de applicatie. Dit komt omdat er een verschil kan zijn in hoe de eigenaars van de applicatie denken dat de applicatie gebruikt wordt en in hoe de gebruikers de applicatie gebruiken. Als er een verschil in deze denkwijze zit, dan zou het design best aangepast worden naar de smaak van de gebruikers. Dit bevordert enerzijds de kwaliteit van de applicatie en anderzijds cre\"eert het een band tussen gebruiker en developer. Indien er geen input komt van de gebruikers kan het nog steeds zijn dat de applicatie niet gebruikt wordt hoe de eigenaar het denkt. Er moet dan ontdekt worden welke elementen gebruikt worden en welke niet. Het monitoren van deze elementen kan ervoor zorgen dat er een goed beeld gevormd wordt voor de eigenaar die met deze informatie zijn inzicht in de applicatie kan veranderen. Dit zorgt ervoor dat de eigenaar betere beslissingen kan maken omtrent de toekomst en de verdere ontwikkeling van de applicatie. Zelfs al zijn er reviews van de gebruikers, dan nog is het voor de eigenaar meestal onmogelijk om te ontdekken welke elementen in de applicatie gebruikt worden en welke niet. Het is dus belangrijk om de applicatie te monitoren, ookal is er veel input van de gebruikers. \\


De prestaties van een applicatie zijn belangrijk voor een eigenaar omdat deze een impact hebben op de gebruiksvriendelijkheid van de applicatie. De meeste problemen die te maken hebben met prestaties zijn op te lossen door voldoende testen te schrijven en hiermee bugs en trage code sequenties uit de code te halen, maar sommige prestatie problemen doen zich enkel voor bij een significant gebruikersaantal. Drie uit de top tien van meest vermelde klachten van mobiele applicaties hebben te maken met de prestaties van een applicatie, namelijk: \textit{Resource-Heavy, Slow or lagging} en \textit{Frequent Crashing}. Deze klachten worden vaak in reviews van gebruikers geuit. De eigenaars kunnen hieruit opmaken dat er een probleem is, maar developers weten niet zeker waar het fout loopt uit die reviews. Om uit te vinden waar de prestatie problemen zitten is het voor de developers handig om de applicatie te monitoren waar de bottlenecks kunnen ontstaan. Hierdoor kunnen ze de exacte oorzaak van het probleem vinden. Op deze manier kan een potenti\"eel probleem al ontdekt worden voor deze in de reviews van de gebruikers opduikt. Dit zorgt ervoor dat de tevredenheid niet naar beneden gaat, omdat het probleem op voorhand al ontdekt wordt. \\
%cite http://appealingstudio.com/why-your-app-sucks-the-20-most-common-complaints-about-mobile-apps-2/


De reviews die door de gebruikers worden gelezen moeten met een korrel zout genomen worden; mensen hebben de intentie om te overdrijven. De eigenaars kunnen hiervoor gebruik maken van een methode, ontwikkeld door de Carnegie Mellon Universiteit, die een analyse uitvoert van deze reviews en de inconsistente reviews eruit filtert. Zo kan er een goed standpunt gevormd worden rond de kwaliteit van de applicatie. Maar zelfs met die methode zijn reviews nog steeds subjectief als het gaat over prestaties. De enige manier om objectief te redeneren hierover is door middel van metingen uit te voeren. Deze metingen kunnen ge\"implementeerd worden door de developer zelf of door een monitoring library. Het gebruik van een monitoring library brengt vele voordelen met zich mee ten opzichte van het zelf implementeren van een systeem. Een monitoring library is ontwikkeld met als doel het zo goed mogelijk monitoren van een applicatie en heeft dus alle diensten hiervoor in huis. De tijd en moeite die een developer in het monitoren van een applicatie steekt is significant lager dan dat de developer zelf nog een systeem moet implementeren. Dit komt omdat de monitoring library klaar is om gebruikt te worden en alle communicatie in deze library verwerkt is, terwijl een developer zelf nog deze communicatie zou moeten schrijven en een back end. Een monitoring library zorgt voor het verwerken en het weergeven van deze data in grafieken zodat er een goed beeld hiervan gevormd kan worden. Een nadeel van een externe monitoring library te gebruiken is dat de data opgeslagen staat op de servers van de externe partij en dat deze in vele gevallen niet opgevraagd kan worden bij die partij. Een ander nadeel is dat de developer de data anders of dieper wil analyseren dan dat gebeurt in de monitoring library. De eigenaars moeten dus een keuze maken tussen het ontwikkelen van een eigen, nieuw, monitoring systeem waar veel tijd en moeite in kruipt en waarvoor voldoende infrastructuur voor moet voorzien worden (back end servers) en het in gebruik nemen van een bestaande monitoring library die misschien net niet genoeg functionaliteit voorziet of de data niet beschikbaar stelt aan de eigenaars van de applicatie.\\
%cite de paper van de universiteit


Het monitoren van mobiele applicaties is anders dan het monitoren van websites, internet applicaties en pc applicaties omwille van een aantal factoren. Een mobiele applicatie draait op een besturingssysteem dat speciaal ontworpen is voor de mobiele telefoons. Android en iOS zijn de twee besturingssystemen die het waard zijn om te vermelden. Een mobiele applicatie wordt dus meestal voor beiden ontwikkeld om een zo groot mogelijk aantal gebruikers te bereiken. Ontwikkelen voor de andere besturingssystemen is meestal niet rendabel voor de moeite en het geld dat erin moet gestoken worden. De monitoring library draait dus maar op twee verschillende systemen, terwijl er tientallen web browsers bestaan waar websites en internet applicaties in draaien. \\
%cite android en ios market share
Indien de applicaties native worden ontwikkeld, dan moet de applicatie een keer voor Android en een keer voor iOS ontwikkeld worden. Dit zorgt ervoor dat er verschillende bugs en/of bottlenecks in de verschillende applicaties kunnen zitten. Het zou verkeerd zijn om gecollecteerde data van de twee applicaties samen te verwerken. \\
Een desktop computer of laptop heeft in de meeste gevallen ofwel een WiFi verbinding of een ethernet verbinding met het internet. De internetverbinding van een smartphone varieert tussen WiFi en een mobiel netwerk (4G, 3G, Edge, ...). Deze connecties hebben een verschillende snelheid en moeten dus anders ge\"interpreteerd worden. \\

Bij het monitoren van mobiele applicaties is het noodzakelijk dat een monitoring library zo weinig mogelijk resources (CPU, batterij, netwerk) gebruikt. Zoals eerder al vermeld is dit \'e\'en van de meest voorkomende klachten bij gebruikers van een mobiele applicatie. Bij pc applicaties wordt hier bijna nooit rekening mee gehouden, omdat de prestaties van de resources hier significant hoger zijn dan bij smartphones. Bij internet applicaties is dit niet van toepassing, omdat dit heel erg af hangt van de implementatie van de browser. \\


Het monitoren van mobiele applicaties is op veel vlakken een handige feature. Enerzijds om als ondersteuning te dienen voor de eigenaars om de gebruikers van de applicatie tevreden te stellen en de applicatie te kunnen verbeteren/veranderen in functie van de gebruiker en hoe de applicatie gebruikt wordt. Anderzijds dient het monitoren van de applicatie als een hulpmiddel voor de developers om te kunnen ontdekken of er bugs in de applicatie zitten en ook de plaats van voorkomen te identificeren zodat deze opgelost kunnen geraken in een volgende versie. Er moet een keuze gemaakt worden tussen het gebruik van een bestaand monitoringsysteem of het zelf ontwikkelen van zo'n systeem met alle voordelen en nadelen in rekening gebracht. \\


\section{Development Scenarios}
In deze sectie worden een aantal scenarios geschetst waar een developer of eigenaar er baat bij heeft om een monitoring library te gebruiken. 

\subsection{General Purpose App}
Een general purpose app is een mobiele applicatie die geen game is (bv. Facebook, Shazam, maar ook een gewone camera app). De functionaliteiten van deze applicaties zijn heel uiteenlopend. Een eigenaar en een developer hebben verschillende informatie nodig om beslissingen omtrent de applicatie te nemen. Deze worden daarom in de volgende paragrafen afgezonderd.
%cite facebook, shazam

\paragraph{Eigenaar}
Een eigenaar wil dat de applicatie zoveel mogelijk gebruikt wordt en zoveel mogelijk geld opbrengt (door bv. advertenties). Om gebruikers te lokken en deze ook te houden moet de eigenaar de applicatie af en toe updaten om gegeerde features toe te voegen of te verbeteren en niet gebruikte features te verwijderen. Zonder een of ander monitorsysteem is het voor de eigenaar bijna onmogelijk om te weten welke features vaak gebruikt worden en welke niet. 

Indien de eigenaar weet welke onderdelen vaak gebruikt worden kan hij hierop inspelen door goed geplaatste advertenties in de applicatie in te bouwen en hiermee geld te verdienen. Het beste zou dan zijn om het onderdeel uit de applicatie te halen. Indien een onderdeel zelden gebruikt wordt is het niet voordelig om dit onderdeel te blijven ondersteunen of verbeteren. Het zou beter zijn om dit uit de applicatie te verwijderen.

De eigenaar kan het design van de applicatie aanpassen om een weinig gebruikt onderdeel meer in de spotlight te zetten zodat gebruikers dit vaker zouden gebruiken. \\


\paragraph{Developer}
De eigenaar wil vooral weten hoe de applicatie gebruikt wordt. De developer wil vooral weten of de applicatie naar behoren werkt en er geen bugs of bottlenecks in zitten. De meeste bugs zouden er in de testfase uitgehaald moeten worden, maar er zijn bugs en bottlenecks die enkel op grote schaal zichtbaar zijn. Meestal heeft dit te maken met het ontvangen of verzenden van content over het netwerk, maar kan ook te maken hebben met het lokaal verwerken van data. Zonder een monitorsysteem is het voor de developers in vele gevallen een hele uitdaging om de bottleneck of de bug te vinden. Met een monitorsysteem kunnen de developers op bepaalde kritieke punten metingen uitvoeren. Indien er dan een bottleneck aanwezig is in de applicatie, dan weten de developers op welk punt het fout loopt. Hierdoor wordt er veel tijd bespaard in het zoeken naar deze bottleneck. \\



\subsection{Game}
De game die besproken wordt is het populaire spel \textbf{Angry Birds}. In dit spel krijgt een speler drie beurten om alle vijanden te verslaan door een constructie omver te werpen en de vijanden schade te berokkenen. Indien alle vijanden verslagen zijn eindigt het level en krijgt de speler een score en een beoordeling van maximaal drie sterren. Dit scenario wordt opgedeeld in een scenario voor de eigenaar en een developer.
%cite angry birds

\paragraph{Eigenaar}
De eigenaar wil ervoor zorgen dat de applicatie zoveel mogelijk gebruikt wordt en dat de levels niet te moeilijk, maar ook niet te gemakkelijk zijn. Om te kijken dat de levels niet te moeilijk zijn kan de eigenaar de developers vragen om per level bij te houden hoe vaak er op de reset toets gedrukt is, hoeveel sterren de spelers gemiddeld verzamelen en hoeveel beurten er gemiddeld gebruikt worden. Hieruit kan de eigenaar dan opmaken of de levels de gewenste moeilijkheidsgraad hebben en in de toekomst deze getallen mee in rekening te nemen in het ontwikkelen van nieuwe levels.

De populariteit van een spel hangt vaak af hoe vaak het gespeeld wordt en hoe lang een spelsessie duurt. Dit kan gemeten worden met een monitoring library door elke keer dat de applicatie gestart wordt dit door te sturen naar de server en een timer te starten die stopt wanneer de gebruiker de app sluit. De eigenaar kan zo ingrijpen als de app minder (lang) gebruikt wordt dan voorheen. \\

\paragraph{Developer}
De developer wil vooral op de hoogte zijn van de prestaties van de applicaties en eventuele bugs vinden. Bij Angry Birds heeft dit impact op twee verschillende situaties, namelijk de periode dat de gebruiker zich in het menu bevindt en de periode dat de gebruiker het spel speelt. Het verschil hierin is wat er gemonitord wordt. In het spel zelf is het belangrijk dat de framerate niet te laag wordt. De developer zal dit dan ook monitoren in het spel zelf. In het menu is dit minder belangrijk. Hier is het eerder belangrijk dat er geen significante vertragingen oplopen in het navigeren door de menus en de verschillende collecties van levels.\\


Deze development scenarios tonen aan dat het gebruik van een monitoring library in mobiele applicaties niet alleen de developer, maar ook de eigenaar kan helpen in het maken van beslissingen omtrent de applicatie. Op deze manier kan er data verzameld worden wanneer de applicatie al in de handen is van de gebruikers. Dit staat tegenover een testomgeving die meestal enkel op kleine schaal uitgevoerd wordt.






